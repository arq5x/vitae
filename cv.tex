%______________________________________________________________________________________________________________________
% @brief    LaTeX2e Resume for Aaron R Quinlan
\documentclass[margin,line]{cv}
\usepackage{url}
\usepackage{enumerate}

%______________________________________________________________________________________________________________________
\begin{document}
\name{\Large Aaron R. Quinlan, Ph.D.}
\begin{resume}
    %__________________________________________________________________________________________________________________
    % Contact Information
    \section{\mysidestyle Contact\\Information}
    Assistant Professor                                                                     \hfill (O): 434.243.1669\\%
    Department of Public Health Sciences                                                    \hfill (F): 434.924.1312\\%
    Center for Public Health Genomics                                                \hfill \url{arq5x@virginia.edu}\\%
    University of Virginia                                                    \hfill \url{cphg.virginia.edu/quinlan}\\%

    %__________________________________________________________________________________________________________________
    % Research Interests
    \section{\mysidestyle Research\\Interests}

    Chromosome stability and somatic genome evolution; Genomics software development; Cancer genomics;
    Population genomics; Genetics of complex disease 

    %__________________________________________________________________________________________________________________
    % Education
    \section{\mysidestyle Education}

    \textbf{Boston College}, Chestnut Hill, MA, USA\\
    Ph.D., Biology, 2008\\
    \\
    \textbf{College of William and Mary}, Williamsburg, VA, USA\\
    B.S., Computer Science, 1997


    %__________________________________________________________________________________________________________________
    % Academic Experience
    \section{\mysidestyle Academic Experience}

    \textbf{University of Virginia}, Charlottesville, VA, USA\\
    \textit{Assistant Professor of Public Health Sciences}                              \hfill 2011-\\
    Center for Public Health Genomics\\
    \\
    \textbf{University of Virginia}, Charlottesville, VA, USA\\
    \textit{NRSA Postdoctoral Fellow (NHGRI)}                                          \hfill 2008-2011\\


    %__________________________________________________________________________________________________________________
    % Honours and Awards
    \section{\mysidestyle Honors and\\Awards} 
    
    Co-chair, NHLBI Exome Sequence Project Structural Variation Group. 2011-                           \\\vspace{1mm}%
    Instructor for CSHL Advanced Sequencing Technologies Course. 2009-                                 \\\vspace{1mm}%
    Fund for Excellence in Science and Technology Awardee, UVa. (1 of 5). 2011                         \\\vspace{1mm}%
    Ruth L. Kirschstein (NRSA / F32) Postdoctoral Fellowship, NHGRI. 2009-2010                         \\\vspace{1mm}%
    Presidential Fellowship, Boston College. 2004-2007                                                               %
    
    
    %__________________________________________________________________________________________________________________
    % Editing and Review
    \section{\mysidestyle Editing and\\Review} 
    
    \textit{Ad hoc} reviewer for \textit{Genome Research}, \textit{Genome Biology}, \textit{Bioinformatics}, \\\vspace{1mm}%
    \textit{BMC Bioinformatics}, \textit{Bioessays}, and \textit{Genes}                                      \\\vspace{1mm}%
    \\
    NIH: Special Emphasis Panel for Computational Analysis of ENCODE (U01) (16-May-2012)                                   %
    
    %__________________________________________________________________________________________________________________
    % Publications
    \section{\mysidestyle Publications}
    $\dagger$\textit{ denotes corresponding author}\\
    $\star$\textit{ denotes joint first authors}

    \begin{list}{*}{}
    \item[16.] Krumm N, Sudmant PH, Ko A, O`Roak BJ, NHLBI Exome Sequencing Project, 
    \textbf{Quinlan AR}, Nickerson DA, Eichler EE. 
    Copy number variation detection and genotyping from exome sequence data. Under review.
    
    \item[15.] Layer R$\dagger$, \textbf{Quinlan AR}$\dagger$, Robins G, Skadron K. 
    Binary Interval Search (BITS): A Massively Parallel Interval Intersection Algorithm. 
    Under review.
    
    \item[14.] \textbf{Quinlan AR} and Hall IM. 
    Structural variation in mammalian genomes. Invited book chapter for Methods in Molecular Biology.
    In press.
   
    \item[13.] \textbf{Quinlan AR}, Boland MJ, Leibowitz ML, Shumilina S, Pehrson SM, Baldwin KK, Hall IM. 
    Paired-end DNA sequencing of induced pluripotent stem cell genomes reveals rare structural mutations and retroelement stability. 
    \textit{Cell Stem Cell}, 9:366-373, 2011.
    
    \item[12.] Keene KL, \textbf{Quinlan AR}, Hou X, Hall IM, Mychaleckyj, Onengut-Gumuscu S, Concannon P.
    Evidence for two independent associations with type 1 diabetes at the 12q13 locus. 
    \textit{Genes and Immunity}, 13:66-70, 2011.
    
    \item[11.] \textbf{Quinlan AR}, Hall IM. 
    The landscape of complex chromosomal rearrangements in the germline and somatic lineages. 
    \textit{Trends in Genetics}, 18:43-53, 2011.
    
    \item[10.] Dale R, Pedersen B, \textbf{Quinlan AR}$\dagger$. 
    Pybedtools: a flexible Python library for manipulating genomic datasets and annotations. 
    \textit{Bioinformatics}, 24:3423-3424, 2011.\\
    \url{packages.python.org/pybedtools/}
    
    \item[9.] Barnett D, Garrison E, \textbf{Quinlan AR}, Stromberg M, Marth G.
    BamTools: a C++ API and toolkit for analyzing and managing BAM files.
    \textit{Bioinformatics}, 12:1691-1692, 2011.\\
    \url{code.google.com/p/bamtools}

    \item[8.] 1000 Genomes Project Consortium.. 
    A map of human genome variation from population-scale sequencing. 
    \textit{Nature} 7319:1061-73, 2010.

    \item[7.] \textbf{Quinlan AR} and Hall IM.
    BEDTools: A flexible framework for comparing genomic features. 
    \textit{Bioinformatics}, 6:841-842, 2010.\\
    \url{code.google.com/p/bedtools}

    \item[6.] \textbf{Quinlan AR}, Clark RA, Sokolova, S, Leibowitx ML, Zhang Y, Hurles ME, Mell JC, Hall IM. 
    Genome-wide mapping and assembly of structural variant breakpoints in the mouse genome. 
    \textit{Genome Research}, 20:623-635, 2010.\\
    \url{code.google.com/p/hydra-sv}

    \item[5.] Sackton, TB, Kulathinal RJ, Bergman CM, Quinlan AR, Dopman E, Marth GT, Hartl DL, Clark AG. 
    Population Genomic Inferences from Sparse High-Throughput Sequencing of Two Populations of Drosophila melanogaster. 
    \textit{Genome Biol Evol}, 1:439-455, 2009.

    \item[4.] Smith D, \textbf{Quinlan AR}$\star$, Peckham HR, \textit{et al}. 
    Rapid whole-genome mutational profiling using next-generation sequencing technologies. 
    \textit{Genome Research}, 18:1638-1642, 2008.

    \item[3.] Hillier LW, Marth GT, \textbf{Quinlan AR}, \textit{et al}. 
    Whole Genome Sequencing and SNP Discovery for C. elegans using massively parallel sequencing-by-synthesis. 
    \textit{Nature Methods}, 5:183-188, 2008.
    
    \item[2.] \textbf{Quinlan AR}, Stewart D, Stromberg M, Marth GT. 
    PyroBayes: Accurate quality scores for 454 Life Science pyrosequences.
    \textit{Nature Methods}, 5:179-181, 2008.
    
    \item[1.] \textbf{Quinlan AR}, Marth GT. 
    Primer-site SNPs mask mutations. 
    \textit{Nature Methods}, 4:192, 2007.

    \end{list}
    

    %__________________________________________________________________________________________________________________
    % Research Support
    \section{\mysidestyle Research Support}

    Project Title: \textit{New algorithms and tools for large-scale genomic analysis.} \\
    PI: Aaron Quinlan \\
    Source: NIH/NHGRI (R21 ES020521-01) \\
    Amount: \$437,112; Period funded: 19-Apr-2012 - 31-Mar-2016
    
    Project Title: \textit{Identification of radiation sensitivity alleles by whole exome sequencing.} \\
    PI: Pat Concannon \\
    Co-investigator: Aaron Quinlan \\
    Source: NIH/NIEHS (R21 ES020521-01) \\
    Amount: \$153,376; Period funded: 19-Aug-2011 - 31-Jul-2013
    
    \vspace{-2mm}
    Project Title: \textit{New oncogenes and regions of genome instability in ovarian cancer.} \\
    PI: Aaron Quinlan \\
    Source: University of Virginia Fund for Excellence in Science and Technology (FEST) \\
    Amount: \$50,000; Period funded: 01-May-2011 - 30-Apr-2012
    
    \vspace{-2mm}
    Project Title: \textit{Defining the genomic architecture of glioblastoma for improved therapy.} \\
    PI: Aaron Quinlan \\
    Source: University of Virginia Cancer Center Pilot Fund\\
    Amount: \$53,125; Period funded: 01-Apr-2011 - 31-Mar-2012
    
    \vspace{-2mm}
    Project Title: \textit{Rates and patterns of recurrent structural variation in the mouse genome.}\\
    PI: Aaron Quinlan\\
    Source: NIH/NHGRI (F32 HG005197-02)\\
    Period funded: 01-Aug-2009 - 31-Dec-2010

    %__________________________________________________________________________________________________________________
    % Seminars
    \section{\mysidestyle Invited Lectures}

    \textit{Towards a map of structural variation in the Exome Sequencing Project.} \\
    NHLBI Exome Sequencing Project In-Person Meeting, March 28, 2012

    \vspace{-2mm}
    \textit{Exploring the origin and extent of structural variation in human genomes.} \\
    Dean’s New Faculty Seminar Series, University of Virginia School of Medicine, Jan 19, 2012

    \vspace{-2mm}
    \textit{ESP Structural Variation Project Group: goals, initial results, and future work.} \\
    NHLBI Exome Sequencing Project In-Person Meeting, June 9, 2011
    
    \vspace{-2mm}
    \textit{Large-Scale Characterization of SV Breakpoints in Cancer.} \\
    Keystone Symposium on The Functional Impact of Structural Variation, Jan. 11, 2011

    \vspace{-2mm}
    \textit{Efficient discovery of structural instability in repetitive regions of mammalian genomes.} \\
    Advances in Genome Biology and Technology, Feb. 2009

    \vspace{-2mm}
    \textit{Approaches to rare allele discovery: More samples or more depth per sample? } \\
    1000 Genomes Analysis Meeting, Cold Spring Harbor Laboratories, May 2008.
    
    %__________________________________________________________________________________________________________________
    % Adademic Service
    %__________________________________________________________________________________________________________________
    \section{\mysidestyle Academic Service}
    
    Member of the Univ. of Virginia Bioinformatics Core Advisory Committee.\\
    01-Nov-2011 - Present.
    
    \vspace{-2mm}
    Served on the Univ. of Virginia Bioinformatics Core Director Search Committee. \\
    Summer 2011.
    %__________________________________________________________________________________________________________________



    %__________________________________________________________________________________________________________________
    % Teaching
    %__________________________________________________________________________________________________________________
    \section{\mysidestyle Teaching}
    Instructor for CSHL Advanced Sequencing Technologies Course.\\
    2009 - Present
    
    \vspace{-2mm}
    Guest Lecturer for graduate course in genomics (BIOCH 5080).\\
    15-Feb-2011
    %__________________________________________________________________________________________________________________
    
    
    
    %__________________________________________________________________________________________________________________
    % Thesis Committees
    %__________________________________________________________________________________________________________________
    \section{\mysidestyle Thesis Committees}
    \textbf{Current} \\
    Michael Lindberg (Ph.D., Biochemistry and Molecular Genetics, qualified 2012)
    %__________________________________________________________________________________________________________________
    
    

%______________________________________________________________________________________________________________________
\end{resume}
\end{document}


%______________________________________________________________________________________________________________________
% EOF

